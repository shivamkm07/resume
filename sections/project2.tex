\cvsection{Projects}
\begin{cventries}
	
	  \cventry
	  {Course Project(CS335), Prof. Swarnendu Biswas}
	  {\href{https://github.com/shivamkm/java-compiler}{Java Compiler} }
	  {\emph{\texttt{\href{https://github.com/shivamkm/java-compiler}{ github.com/shivamkm/java-compiler}}}}
	  {Jan'20 - Apr'20}
	  {
	    \begin{cvitems}
	      \item Designed \textbf{lexer} and \textbf{parser} of a java compiler using \textbf{PLY} framework, printing Abstract Syntax Tree(AST) as output
	      \item Added support for the \textbf{symbol table} structure
	      % using it further for limited error checking i.e. scope and type errors
	      \item Extended the compiler to generate \textbf{3-address code(3AC)}
	    %   if the program is syntactically and semantically correct
	      \item Provided support for 
	      functions, classes, interfaces etc.
	    %  limited static and dynamic polymorphism, primitive typecasting etc.
	    \end{cvitems}
	  }
	
	
	\cventry
	{Course Project(CS641), Prof. Manindra Agrawal}
	{\href{https://github.com/shivamkm/decipher}{Cipher Decoder}}
	{\emph{\texttt{\href{https://github.com/shivamkm/decipher}{github.com/shivamkm/decipher}}}}
	{Jan'20 - Apr'20}
	{
		\begin{cvitems}
			\item Implemented decryption algorithms for multiple ciphers including \textbf{Caesar, Permutation-Substitution, Vigenere}
		%	\item Used Hill Climbing Algorithm for cipher decryption using suitable termination condition and improvement heuristic
			\item Implemented Differential Cryptanalysis of Data Encryption Standard(\textbf{3-DES}) assuming standard key scheduling
		\end{cvitems}
	}
\cventry
{Course Project(CS330), Prof. Debadatta Mishra}
{\href{https://github.com/shivamkm/gemOS}{Building GemOS}}
{\emph{\texttt{\href{https://github.com/shivamkm/gemOS}{github.com/shivamkm/gemOS}}}}
{Aug'19 - Nov'19}
{
	\begin{cvitems}
		\item Implemented file system calls like open(), read(), write() etc.
		% pipe(), fork(), lseek() etc.
		\item Implemented mmap(), munmap() and mprotect(), while handling \textbf{lazy allocation} and \textbf{pagefaults}
		\item Implemented syscalls like cfork() and vfork(), taking care of \textbf{copy-on-write} mechanism on shared memory regions
	\end{cvitems}
}
\cventry
{Course Project(CS771), Prof. Purushottam Kar}
{\href{https://github.com/shivamkm/machine-learning}{Machine Learning}}
{\emph{\texttt{\href{https://github.com/shivamkm/machine-learning}{github.com/shivamkm/machine-learning}}}}
{Aug'19 - Nov'19}
{
	\begin{cvitems}
		\item Employed algorithms like \textbf{SGD}, Coordinate Maximisation, Coordinate Descent etc. for a binary classification problem
		\item Implemented a \textbf{CNN} with linear layers to solve the given image classification problem using \textbf{Keras}
		\item Built a recommendation system using \textbf{multi-label} classifier Bonsai with suitable changes to reduce the time overhead
	%	\item Executed Bonsai for a multi-label classification problem
	%	 with suitable modifications to reduce the time overhead
	\end{cvitems}
}
\cventry
{Course Project(CS252), Prof. Nisheeth Srivastava}
{\href{https://github.com/shivamkm/mobile-app}{Mobile App}}
{\emph{\texttt{\href{https://github.com/shivamkm/mobile-app}{github.com/shivamkm/mobile-app}}}}
{Aug'19 - Nov'19}
{
	\begin{cvitems}
		\item Built a fully-functional \textbf{MERN} application with Secure Login Management Protocol
		\item Employed MongoDB, Express.js with Node.js on server-side and React-native on client side
		\item Used mobile-native functionalities like camera and gallery
		% to click and upload pictures
	\end{cvitems}
}


\cventry
{Course Project(CS202), Prof. Subhajit Roy}
{\href{https://github.com/shivamkm/sat-solver}{SAT Solver}}
{\emph{\texttt{\href{https://github.com/shivamkm/sat-solver}{github.com/shivamkm/sat-solver}}}}
{Aug'18 - Nov'18}
{
	\begin{cvitems}
		\item Implemented a SAT Solver for propositional logic in python using Davis Putnam Logemann Loveland (\textbf{DPLL)} algorithm
	%	\item Employed several heuristics for optimizing the time overhead of SAT solver, making it competitive to MINISAT
		\item Encoded \textbf{diagonal sudoku} problem in DIMACS form using propositional logic and solved it using self-coded SAT solver
	\end{cvitems}
}
\end{cventries}
