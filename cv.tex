%!TEX TS-program = xelatex
%!TEX encoding = UTF-8 Unicode
% Awesome CV LaTeX Template for CV/Resume
%
% This template has been downloaded from:
% https://github.com/posquit0/Awesome-CV
%
% Author:
% Claud D. Park <posquit0.bj@gmail.com>
% http://www.posquit0.com
%
% Template license:
% CC BY-SA 4.0 (https://creativecommons.org/licenses/by-sa/4.0/)
%


\documentclass[10pt, a4paper]{awesome-cv}
\geometry{left=1.4cm, top=.8cm, right=1.4cm, bottom=1.8cm, footskip=.5cm}
\fontdir[fonts/]

% Color for highlights
% Awesome Colors: awesome-emerald, awesome-skyblue, awesome-red, awesome-pink, awesome-orange
%                 awesome-nephritis, awesome-concrete, awesome-darknight
\colorlet{awesome}{awesome-darknight}
% Uncomment if you would like to specify your own color
% \definecolor{awesome}{HTML}{CA63A8}

% Colors for text
% Uncomment if you would like to specify your own color
% \definecolor{darktext}{HTML}{414141}
% \definecolor{text}{HTML}{333333}
% \definecolor{graytext}{HTML}{5D5D5D}
% \definecolor{lighttext}{HTML}{999999}

% Set false if you don't want to highlight section with awesome color
\setbool{acvSectionColorHighlight}{false}

% If you would like to change the social information separator from a pipe (|) to something else
\renewcommand{\acvHeaderSocialSep}{\quad\textbar\quad}


% Available options: circle|rectangle,edge/noedge,left/right
% \photo{./profile.png}
\name{Yash}{Srivastav}
\position{Senior Undergraduate{\enskip\cdotp\enskip}Computer Science and Engineering}
\address{Indian Institute of Technology, Kanpur}
\mobile{(+91) 705-413-3662}
\email{yash111998@gmail.com}
\homepage{yashsriv.org}
\github{yashsriv}
\linkedin{yashsriv}
% \twitter{@therealyashsriv}
% \quote{``There is no fate but what we make."}

\newcommand{\smallcventry}[6]{\cventry{#1}{#2}{#3}{#4}{#6}}
\newcommand{\specialcvsection}[1]{\cvsection{#1}}




\begin{document}
\makecvheader
\makecvfooter
  {}
  {}
  {\thepage}

\specialcvsection{Educational Qualifications}

\newcommand{\education}[4]{
  & #1 & #2 & &#3 & #4
}
\begin{center}
\begin{tabular}{ | L{0.05cm} l | L{3cm} | L{0.05cm} C{8cm} | r |}
  \hline
  \education{\textbf{Year}}{\textbf{Degree}}{\textbf{Institution(Board)}}{\textbf{CGPA/\%}}\\
  \hline
  \education{2021}{B.Tech, CSE}{Indian Institute of Technology, Kanpur}{9.4/10.0}\\
  \education{2017}{AISSCE -- XII}{Sunbeam English School Bhagwanpur (CBSE)}{96.4\%}\\
  \education{2015}{AISSE -- X}{Pristine Children's High School (CBSE)}{10.0/10.0}\\
  \hline
\end{tabular}
\end{center}
\vspace{-4mm}

%%% Local Variables:
%%% mode: latex
%%% TeX-master: "../cv.tex"
%%% TeX-engine: xelatex
%%% End:
\cvsection{Scholastic Achievements}
\begin{cvhonors}

  \cvhonor
  {\ifdefined \ONEPAGE \else All India \fi \textbf{Academic Excellence Award}}
  {awarded twice for outstanding academic performance in year '17-18 and '18-19}
  {}
  {2015}

  \cvhonor
  {\ifdefined \ONEPAGE \else All India \fi  \textbf{Ram Prakash Chopra Memorial Scholarship}}
  {awarded for exceptional academic record as a sophomore }
  {}
  {2015}

  \cvhonor
  {\textbf{All India Rank 348}}
  {JEE Mains}
  {}
  {2015}
  
  \cvhonor
	{\textbf{All India Rank 715}}
	{JEE Advanced}
	{}
	{2015}
	
\cvhonor{\textbf{KVPY Scholarship Awardee}}{amongst 50000 candidates}{arg3}{arg4}
\cvhonor{\textbf{Top 1\%}}{National Standard Examination in Physics, U.P.}{arg3}{arg4}
\end{cvhonors}

%%% Local Variables:
%%% mode: latex
%%% TeX-engine: xetex
%%% TeX-master: "../cv"
%%% End:
\cvsection{Work Experience}
\begin{cventries}

  \cventry
  {Quantitative Research Intern}
  {E-Trading Team, JP Morgan \& Chase}
  {Mumbai (Remote)}
  {May'20 - Jul'20}
  {
    \begin{cvitems}
      \item Employed several Univariate \textbf{feature selection tests} for the analysis of existing baseline price prediction model
      \item Used feedforward and recurrent neural networks (\textbf{LSTM}), linear as well as non-linear \textbf{regression} to improve accuracy
      \item	\textbf{Standardized} data for optimizing parameterization of individual features, yielding significant improvement
      \item Implemented several L1 as well as L2 features
      \item Improved model predictive power by \textbf{40\%} for HK names
    \end{cvitems}
  }
\end{cventries}
\vspace{-2mm}

%%% Local Variables:
%%% mode: latex
%%% TeX-engine: xetex
%%% TeX-master: "../cv.tex"
%%% End:
\vspace{-0.2cm}
\cvsection{Skills}
\ifdefined\ONEPAGE
\\
\vspace{-0.1cm}
\textbf{Languages}: C/C++, Python, Haskell, Java, JavaScript, PHP\\
%\textbf{Experienced}: C++, Java, Scala, Android\\
%\textbf{Exposure}: Haskell, Rust, Dart, Perl\\
%\textbf{Web}: Angular, Akka, TypeScript, Redux, Flutter\\
\vspace{0.05cm}
\textbf{Utilities}: Linux, Git, \LaTeX, MySql, Kubernetes, Dapr, LLVM
% , Numpy, Pandas
%  MongoDB, Numpy, Pandas, Tensorflow, 

%\LaTeX, Vim, Emacs, Vagrant

\else
\begin{cvskills}

  \cvskill
  {Proficient}
  {C, Golang, Python, Javascript}

  \cvskill
  {Experienced}
  {C++, Java, Scala, Android}
  
  \cvskill
  {Exposure}
  {Haskell, Rust, Dart, Perl}
  
  \cvskill
  {Frameworks}
  {Express.js with Node.js, Akka with Scala, JavaScript, TypeScript, Angular,
    Redux, Flutter}

  \cvskill
  {Utilities}
  {Linux shell utilities, Git, Docker, Ansible, Postgres,
    MongoDB, OpenCV, \LaTeX, Vim, Emacs, vagrant}

\end{cvskills}
\fi
%%% Local Variables:
%%% mode: latex
%%% End:
\cvsection{Relevant Courses}

\ifdefined\ONEPAGE

% \textbf{CS:} Introduction to Programming(A$*$), Logic in Computer
% Science, Computer Organization, Data Structures and Algorithms, Computing
% Laboratories - 1(A$*$)
\begin{tabular*}{\textwidth}{l l l l}
  Operating Systems & Advanced Algorithms  & Machine Learning &   Advanced Computer Architecture   \\
  Compiler Design & Data Structures and Algorithms & Modern Cryptology & Programming for Performance\\
  Database Systems & Discrete Mathematics & Theory of Computation & Parallel Computing\\
  Computer Organization &  Statistical Natural Language Processing & Comp. LabI (Bash+Haskell) &  Comp. LabII (LAMP+MERN)\\
\end{tabular*}



% \textbf{Math}: Discrete Math, Probability and Statistics(A$*$)


{\small
%    {\hfill  $i$: In progress}
}

\else
{\fontsize{11pt}{1em}\bodyfontlight\upshape\color{text}
  \begin{tabular*}{\textwidth}{l l l}
    Introduction to Programming(A$*$) & Discrete Mathematics  & Computer Organization \\
    Computer Architecture & Data Structures and Algorithms & Probability \& Statistics(A$*$) \\ 
    Computing Laboratories - 1(A$*$) & Computing Laboratories - 2(A$*$) & Compiler Design \\
    Functional Programming(A$*$) & Computer Systems Security & Computer Networks($i$)
  \end{tabular*}
}
{\fontsize{11pt}{1em}\footerfont\upshape\color{text}
  \begin{tabular*}{\textwidth}{ l l }
    \entrylocationstyle{A$*$: Grade for exceptional performance} & \entrylocationstyle{$i$: In progress}\\
  \end{tabular*}
}
\vspace{-0.5cm}

\fi

%%% Local Variables:
%%% mode: latex
%%% TeX-engine: xetex
%%% TeX-master: "../cv"
%%% End:
\newpage
\cvsection{Research Experience}

\begin{cventries}

  \cventry
  {Supervisor: Prof. Swarnendu Biswas}
  {Data Race Detection, Task-Parallel Programs}
  {IIT Kanpur}
  {Jun'19 - Aug'21}
  {
    \begin{cvitems}
 \item Implemented SOTA algorithm FastTrack for Task Parallel Programs, using \textbf{LLVM} pass for memory instrumentation
 \item Created an \textbf{optimized} form of FastTrack called FastRacer, reducing space as well as time complexity of detector
  % of metadata operations making it execute on all real-world benchmarks
 \item Designed  \textbf{novel} algorithm Tasker by
 integrating 
 \textbf{vector clocks}
  with space efficient
%  a popular
 tree-based techniques
%  PTracer
%  with the cheap
  % data race checking of
   
%  -based FastRacer
 \item FastRacer achieved speedup of \textbf{1.46X} and Tasker \textbf{1.48X} on 128GB-Intel Xeon system with PTRacer as baseline
 
% and Tasker outperformed PTRacer by \textbf{1.46X}
% 
% Outperformed PTRacer by \textbf{1.5X} on a 128GB-Intel Xeon system, 1.3X on two another Intel Xeon and I9-9900 systems
%  \item Paper based on the research work is under review
    \end{cvitems}
  }

\end{cventries}
%%% Local Variables:
%%% mode: latex
%%% TeX-master: "../cv.tex"
%%% TeX-engine: xelatex
%%% End:
\cvsection{Positions of Responsibility}



  \textbf{Academic Mentor},\small\emph{ Counselling Service, IIT Kanpur, 2018-19}\\
%    \begin{itemize}
%  \item \normalsize Provided academic assistance to needy students through remedial classes and One-to-one mentoring\\
%\end{itemize}
\vspace{0.05cm}
  % in decisions regarding the festival and was responsible for managing the stays
  % and travels of all Celebrities and Artists invited to the event.
 \normalsize \textbf{Secretary},\small\emph{ Dramatics Club, IIT Kanpur, 2018-19}
%    \begin{itemize}
%    %Guided freshmen manoeuvring the subtle nuances of theatrics
%  \item \normalsize Participated and performed in several street plays, stage plays and mimes staged in the institute
%   \end{itemize}
  % initiative of conducting a \textbf{``Winter Camp''} where a select few
  % freshmen were introduced to various topics.
  % ranging from cryptography to web development.

% \item \textbf{Secretary}, \emph{Programming Club, IIT Kanpur 2016-17}
%   \ifdefined\ONEPAGE
%   \else
%   : \\
%   Helped Conduct and organize various lectures for freshmen as well as developed
%   a few web applications under the programming club.
%   \fi
% \item \textbf{Senior Executive, Web}, \emph{Antaragni 2016}
%   \ifdefined\ONEPAGE
%   \else
%   : \\
%   Worked on a NodeJS webserver for a college fest. Had a dynamic website
%   modifiable easily by non-programmers and supported android app as well with an
%   API.
%   \fi
\vspace{-2mm}

\cvsection{Miscellaneous}

\begin{itemize}
  % \item Developed a Python Application using \textbf{Pygame} 
  %   \ifdefined \ONEPAGE . \else
  %   for 2 player as well as
  %   single player Reversi gameplay as part of ACA Semester Project.
  %   Link -
  %   \href{https://github.com/yashsriv/Reversi-Python}{github://yashsriv/Reversi-Python}
  %   \fi
  %   \vspace{-1mm}
  % \item Ported the educational OS, nachos, to golang. Link -
  %   \href{https://github.com/yashsriv/go-nachos}{github:yashsriv/go-nachos}
  %   \vspace{-1mm}
  % \item Developed an AI for complete-knowledge two-player games in Haskell as a
  %   course project.
  %   \ifdefined \ONEPAGE \else
  %   Implemented Connect 4 with GUI as an instance of that AI.
  %   Link - \href{https://github.com/yashsriv/haskell-connect-4}{github://yashsriv/haskell-connect-4}
  %   \fi
  %   \vspace{-1mm}
  \item Expoited and patched the zoobar server as part of Computer Systems
    Security Course
  \item Developed an android app which was a Websocket Client for a
    Websocket Server hosting a multiplayer game
  \item Contribute to Open Source projects like pdf.js and thelounge
  \item Won Fresher's Science Quiz in inter-hall annual competition
  \item Among the top 15 teams of India in CSAW 2016, CTF
  \item Mentored 6 students in building a chat application
    \ifdefined \ONEPAGE \else
    using nodejs and
    websockets as an Semester Project
    \fi
    \vspace{-1mm}
\end{itemize}
% \cvsection{Interests}

{\fontsize{11pt}{1em}\bodyfontlight\upshape\color{text}
  \begin{itemize}
  \item Open Source
  \item Capture The Flag Contests
  \item Web Development
  \item Image Processing
  \item Artificial Intelligence
  \item Robotics
  \end{itemize}
}

%%% Local Variables:
%%% mode: latex
%%% TeX-engine: xetex
%%% TeX-master: "../cv"
%%% End:


\end{document}

%%% Local Variables:
%%% mode: latex
%%% TeX-engine: xetex
%%% End: